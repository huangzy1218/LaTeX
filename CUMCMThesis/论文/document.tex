\documentclass{cumcmthesis}
%\documentclass[withoutpreface,bwprint]{cumcmthesis} %去掉封面与编号页
\title{论文题目}
\tihao{A}            % 题号
\baominghao{4321}    % 报名号
\schoolname{你的大学}
\membera{成员A}
\memberb{成员B}
\memberc{成员C}
\supervisor{指导老师}
\yearinput{2017}     % 年
\monthinput{08}      % 月
\dayinput{22}        % 日

\begin{document}
	\maketitle
	\begin{abstract}
		摘要的具体内容。
		\keywords{关键词1\quad  关键词2\quad   关键词3}
	\end{abstract}
	\tableofcontents
	\section{问题重述}
	\subsection{问题背景}
	\subsection{问题重述}
	\section{模型的假设}
	\section{符号说明}
	\begin{center}
		\begin{tabular}{cc}
			\hline
			\makebox[0.3\textwidth][c]{符号}	&  \makebox[0.4\textwidth][c]{意义} \\ \hline
			D	    & 木条宽度(cm) \\ \hline
		\end{tabular}
	\end{center}
	\section{模型的建立与求解}
	\subsection{问题一模型的建立与求解}
	\subsection{问题二模型的建立与求解}
	\subsection{问题三模型的建立与求解}
	\section{模型的检验}
	\section{模型的评价}
	\subsection{模型的优点}
	\subsection{模型的缺点}
	\subsection{模型的推广}
	\begin{thebibliography}{9}%宽度9
		\bibitem[1]{yingyong}
		\newblock MATLAB在数学建模中的应用 \allowbreak[J].
		\newblock 北京航空航天大学出版社, 北京, 2014. 
	\end{thebibliography}
	\begin{appendices}
		\section{排队算法--matlab 源程序}
		
		\begin{lstlisting}[language=matlab]
			kk=2;[mdd,ndd]=size(dd);
			while ~isempty(V)
			[tmpd,j]=min(W(i,V));tmpj=V(j);
			for k=2:ndd
			[tmp1,jj]=min(dd(1,k)+W(dd(2,k),V));
			tmp2=V(jj);tt(k-1,:)=[tmp1,tmp2,jj];
			end
			tmp=[tmpd,tmpj,j;tt];[tmp3,tmp4]=min(tmp(:,1));
			if tmp3==tmpd, ss(1:2,kk)=[i;tmp(tmp4,2)];
			else,tmp5=find(ss(:,tmp4)~=0);tmp6=length(tmp5);
			if dd(2,tmp4)==ss(tmp6,tmp4)
			ss(1:tmp6+1,kk)=[ss(tmp5,tmp4);tmp(tmp4,2)];
			else, ss(1:3,kk)=[i;dd(2,tmp4);tmp(tmp4,2)];
			end;end
			dd=[dd,[tmp3;tmp(tmp4,2)]];V(tmp(tmp4,3))=[];
			[mdd,ndd]=size(dd);kk=kk+1;
			end; S=ss; D=dd(1,:);
		\end{lstlisting}
	\end{appendices}
\end{document}