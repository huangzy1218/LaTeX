\documentclass[12pt]{article}
\usepackage{ctex}
\usepackage[hmargin=1.25in,vmargin=1in]{geometry}%设置页边距
\usepackage{amsmath} %数学符号宏包
\usepackage{graphicx}
\title{\textbf{数学公式教程}}
\author{黄正阳}

\begin{document}
	
	\maketitle
	
	\tableofcontents
	
	\section{行内公式}
	欢迎使用 Slager。
	
	%----------------------------行内公式---------------------------
	当$L_1\tan \alpha \le h \le L_2\tan \alpha$时,我们使用第一种方法。
	
	若$a=b$,则$a+c=b+c$。
	
	%----------------------------行间公式---------------------------
	\section{行间公式}
	如图1所示:
	\begin{equation} %\begin{equation}、
		v_{in}(t) =
		\begin{cases}
			CA\sqrt{\frac{2\varDelta P}{\rho(t)}}, & A\text{处单项阀开启}\\
			0, & A\text{处单项阀关闭}\\
		\end{cases} 
	\end{equation}
	
	如\ref{速度公式}所示:
	\begin{equation} %\begin{equation}、
		v_{in}(t) =
		\begin{cases}
			CA\sqrt{\frac{2\varDelta P}{\rho(t)}}, & A\text{处单项阀开启}\\
			0, & A\text{处单项阀关闭}\\
		\end{cases} 
		\label{速度公式}
		\tag{公式1}
	\end{equation}
	
	如\eqref{速度公式}所示:
	\begin{equation} %\begin{equation}、
		v_{in}(t) =
		\begin{cases}
			CA\sqrt{\frac{2\varDelta P}{\rho(t)}}, & A\text{处单项阀开启}\\
			0, & A\text{处单项阀关闭}\\
		\end{cases} 
		\label{速度公式}
		\tag{公式1}
	\end{equation}
	
	%-----------------一般的数学符号---------------------------
	\section{一般的数学符号(希腊字母、特殊符号)}
	\noindent
	$\alpha \quad \beta \quad \theta \quad \pi \quad \gamma \quad \delta\\
	\Gamma \quad \Delta \quad \Omega\\
	\varOmega \quad \varPhi$
	%\, \; \quad \qquad 
	%-----------------指数、上标下标和导数---------------------------
	\section{指数、上标下标和导数} 
	%上标^ 下表_ 一个字符以上要用{}
	\noindent
	$a_1^2+b_1^2=c_1^2\\
	p_{ij}^3 \quad m_\mathrm{Knuth} \quad \sum_{k=1}^3k\\
	a^x+y \ne a^{x+y} \quad e^{x^2} \ne e^{x2}\\
	f(x)=x^2 \quad f'(x)=2x \quad f''^2(x)=4$
	
	%-----------------分式和根式---------------------------
	\section{分式和根式}
	%\frac{分子}{分母}
	$\frac{1}{2} \quad 3\frac{2}{5}\\$
	
	每节课时长为:$1\frac{1}{2}$ 小时 \qquad $1\dfrac{1}{2}$ 小时
	\[\frac{2}{5} \quad \tfrac{2}{5}\]\\
	%平方根\sqrt{} n次根 \sqrt[n]{}
	$\sqrt{x},\sqrt[3]{2},\sqrt{x^2+\sqrt{y}} $
	
	%-----------------关系符(大于等于号)---------------------------
	\section{关系符(大于等于号)}
	\noindent
	=,>,<,$\ne,\ge,\le,\approx,\sim\\
	A\subset B,A\supset B,a\in A,A\subseteq B,A\supseteq B,\quad \not= $
	
	
	%-----------------算符---------------------------
	\section{算符(加减乘除、正弦余弦函数、e指数、log函数,求极限)}
	\noindent
	+,-,*,/ \\
	$\times \quad \div \quad \cdot,\bullet \quad \pm \quad \mp \\
	\cup \quad \cap \quad \star \\
	\nabla \quad \partial$
	$\sin x \quad \cos \alpha \quad \tan \beta \quad \cot x \quad \sec x\\
	\exp(-2) \quad \log_2{x} \quad \ln{y}\\
	\lim\limits_{x \to \infty} \frac{1}{x}=0$
	
	
	%-----------------巨算符(求积分、求和)---------------------------
	\section{巨算符(求积分、求和)}
	%积分\int 闭曲线求积分\oint 二重积分\iint 三重积分\iiint 求和\sum  求积 \prod
	\noindent
	$\int_0^{\frac{\pi}{2}} \\
	\sum\limits_{i=1}^n$
	
	
	%-----------------数学重音和上下括号---------------------------
	\section{数学重音和上下括号(向量、求导)}
	\noindent
	$\dot{c} \quad \ddot{c} \quad 0.\dot{3}=\dfrac{1}{3}\\
	\vec{r} \quad \hat{\mathbf{e}}\\$
	%使用时要注意重音符号的作用区域,一般应当对某个符号而不是“符号加下标”使用重音:
	$\bar{x}_0 \quad \vec{x}_0\\
	\widehat{AB},\overrightarrow{AB},\underline{3},\overline{2}$\\
	%\overbrace \underbrace
	$\underbrace{\overbrace{(a+b+c)}^6 \cdot \overbrace{(d+e+f)}^7}_\text{meaning of life}=42$
	
	
	%-----------------箭头---------------------------
	\section{箭头}
	\noindent
	$\rightarrow = \to, \leftarrow = \gets\\$
	\[
	a\xleftarrow{x+y+z} b\]
	\[c\xrightarrow[x<y]{a*b*c} d\]
	
	%-----------------括号和定界符---------------------------
	\section{括号和定界符}
	\noindent
	() [] $\{ \}, \langle, \rangle$
	%使用 \left 和 \right 命令可令括号(定界符)的大小可变,自动适应公式,在行间公式中常用。\left 和 \right 必须成对使用。需要使用单个定界符时,另一个定界符写成 \left. 或 \right.
	\[1 + (\frac{1}{1-x^{2}})^3 \qquad
	\frac{\partial f}{\partial t}|_{t=0}\]
	
	\[1 + \left(\frac{1}{1-x^{2}}\right)^3 \qquad
	\left.\frac{\partial f}{\partial t}\right|_{t=0}\]
	
	%-----------------一般的数学符号---------------------------
	\section{复杂公式}
	\[ f(x)=
	\frac{1}{\sqrt{2\pi}\sigma}\mathrm{e}^{-\frac{(x-\mu)^2}{2 \sigma^2}}
	\]
	
	\[ f(x)=
	\frac{1}{\sqrt{2\pi}\sigma}\exp \left(-\frac{(x-\mu)^2}{2 \sigma^2}\right )
	\]
	
	\[ \lim\limits_{N \rightarrow \infty} P \left\{\left|\frac{I(\alpha_i)}{N}-H(s)\right|<\varepsilon\right\}=1 \]
	
	\[ x(n)=\frac{1}{2\pi}\int_{-\pi}^{\pi}{X(\mathrm{e}^ \mathrm{j\omega})\mathrm{e}^ {\mathrm{j}\omega n}}\, \mathrm{d}\omega
	\]
\end{document}
